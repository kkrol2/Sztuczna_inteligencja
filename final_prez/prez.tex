
              
\documentclass{beamer}
%
% Choose how your presentation looks.
%
% For more themes, color themes and font themes, see:
% http://deic.uab.es/~iblanes/beamer_gallery/index_by_theme.html
%
\mode<presentation>
{
  \usetheme{default}      % or try Darmstadt, Madrid, Warsaw, ...
  \usecolortheme{default} % or try albatross, beaver, crane, ...
  \usefonttheme{default}  % or try serif, structurebold, ...
  \setbeamertemplate{navigation symbols}{}
  \setbeamertemplate{caption}[numbered]
} 

\usepackage{polski}

\usepackage[utf8]{inputenc}
\usepackage[T1]{fontenc}
\usepackage{amsfonts}
\usepackage[]{algorithm2e}
\usepackage{lmodern}

\title{Problem podziału na 3 podzbiory}
\author{
       Krzysztof Król\and
                Szymon Dudycz
}
\date{\today}

\begin{document}

\section{Wstęp}
\begin{frame}
  \titlepage
\end{frame}

% Uncomment these lines for an automatically generated outline.
%\begin{frame}{Outline}
%  \tableofcontents
%\end{frame}

\section{Opis problemu}

\begin{frame}{Opis problemu}

Zbiór liczb należy podzielić na 3 podzbiory tak, aby:
\begin{itemize}  
\item każda liczba należała do dokładnie jednego podzbioru
\item sumy liczb elementów każdego podzbioru jak najmniej się różniły
\end{itemize}


\end{frame}  


\section{Dolne ograniczenie}

\begin{frame}[t]{Dolne ograniczenie}

Rozważmy $4$ największe przedmioty $a_1, a_2, a_3$ i $a_4$. Wśród nich są $2$ takie, które będą w tym samym zbiorze, więc jeden ze zbiorów będzie miał wagę co najmniej taką jak suma wag $2$ najlżejszych przedmiotów, powiedzmy $a_3$ i $a_4$.

Różnica między $a_3+a_4$ a sumą wszystkich liczb podzielonych przez $3$ daje nam dolne ograniczenie.

\pause

Zamiast rozważać $4$ największe przedmioty możemy rozważać $k$ największych przedmiotów i wybierać $\lceil \frac{k}{3} \rceil$ z nich. Największa różnica da nam najlepsze dolne ograniczenie.

\end{frame}

\section{Algorytm}

\begin{frame}{Algorytmy}


\begin{itemize}
\item Heurystyka Karmakar-Karp
\item Meta-heurystyka - symulowane wyżarzanie
\end{itemize}

\end{frame}


\end{document}