
              
\documentclass{beamer}
%
% Choose how your presentation looks.
%
% For more themes, color themes and font themes, see:
% http://deic.uab.es/~iblanes/beamer_gallery/index_by_theme.html
%
\mode<presentation>
{
  \usetheme{default}      % or try Darmstadt, Madrid, Warsaw, ...
  \usecolortheme{default} % or try albatross, beaver, crane, ...
  \usefonttheme{default}  % or try serif, structurebold, ...
  \setbeamertemplate{navigation symbols}{}
  \setbeamertemplate{caption}[numbered]
} 

\usepackage{polski}

\usepackage[utf8]{inputenc}
\usepackage[T1]{fontenc}
\usepackage{amsfonts}
\usepackage[]{algorithm2e}
\usepackage{lmodern}

\title{Problem podziału na 3 podzbiory}
\author{
       Krzysztof Król\and
                Szymon Dudycz
}
\date{\today}

\begin{document}

\section{Wstęp}
\begin{frame}
  \titlepage
\end{frame}

% Uncomment these lines for an automatically generated outline.
%\begin{frame}{Outline}
%  \tableofcontents
%\end{frame}

\section{Opis problemu}

\begin{frame}{Opis problemu}

Zbiór liczb należy podzielić na 3 podzbiory tak, aby:
\begin{itemize}  
\item każda liczba należała do dokładnie jednego podzbioru
\item sumy liczb elementów każdego podzbioru jak najmniej się różniły
\end{itemize}

\end{frame}  

\section{Algorytmy}

\begin{frame}{Algorytmy}
\begin{itemize}
\item Heurystyka dla Longest Processing Time
\item Heurystyka Karmakar-Karp
\item Meta-heurystyka - symulowane wyżarzanie
\end{itemize}
\end{frame}

\begin{frame}{Longest Processing Time}
Górne ograniczenie można wyznaczyć algorytmem Longest Processing Time. Można udowodnić, że ten algorytm osiąga górne ograniczenie nie gorsze od $11/9*optimal$ dla problemu podziału ciągu, w którym chcemy minimalizować sumę elementów największego podzbioru.\\~\\
\textbf{LPT:}\\
\begin{algorithm}[H]
 $s_1 \gets \emptyset$\;
 $s_2 \gets \emptyset$\;
 $s_3 \gets \emptyset$\;
 $m \gets $ posortowany ciąg wejściowy w nierosnącej kolejności\;
 $iter \gets 0$\;
 \While{$iter < n$}{
    $s \gets min(s_1,s_2,s_3)$\;
    $s \gets s \cup \{m_{iter}\}$\;
    $iter \gets iter+1$\;
  }
\end{algorithm} 
\end{frame}

\begin{frame}{Symulowane Wyżarzanie}
  \begin{itemize}
    \item Jako sąsiadów obecnego stanu rozważamy te podziały, które można osiągnąć przenosząc jeden element.
    \item Prawdopodobieństwo przejścia do sąsiedniego stanu to $e^{\frac{(c_{old} - c_{new})}{T}}$
    \item \textasciitilde 160000 kroków
  \end{itemize}
  \end{frame}

\section{Dolne ograniczenie}

\begin{frame}[t]{Dolne ograniczenie}

Rozważmy $4$ największe przedmioty $a_1, a_2, a_3$ i $a_4$. Wśród nich są $2$ takie, które będą w tym samym zbiorze, więc jeden ze zbiorów będzie miał wagę co najmniej taką jak suma wag $2$ najlżejszych przedmiotów, powiedzmy $a_3$ i $a_4$.

Różnica między $a_3+a_4$ a sumą wszystkich liczb podzielonych przez $3$ daje nam dolne ograniczenie.

\pause

Zamiast rozważać $4$ największe przedmioty możemy rozważać $k$ największych przedmiotów i wybierać $\lceil \frac{k}{3} \rceil$ z nich. Największa różnica da nam najlepsze dolne ograniczenie.

\end{frame}

\section{Eksperymenty obliczeniowe}

\begin{frame}{Eksperymenty obliczeniowe}
  \begin{tabular}{|c|c|c|c|c|}
  \hline
  $n$  & Dolne ogr. & Karamar-Karp & Wyżarzanie & LPT \\
  \hline
  10 & 38 & 1315 & 598 & 2305 \\
  \hline
  20 & 0 & 238 & 116 & 1318 \\
  \hline
  50 & 0 & 16 & 78 & 617 \\
  \hline
  100 & 0 & 2 & 68 & 318 \\
  \hline
  200 & 0 & 1 & 69 & 283 \\
  \hline
  500 & 0 & 1 & 55 & 218 \\
  \hline
  1000 & 0 & 1 & 42 & 200 \\
  \hline
  2000 & 0 & 1 & 46 & 205 \\
  \hline
  \end{tabular}
\end{frame}


\end{document}