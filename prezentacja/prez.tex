
              
\documentclass{beamer}
%
% Choose how your presentation looks.
%
% For more themes, color themes and font themes, see:
% http://deic.uab.es/~iblanes/beamer_gallery/index_by_theme.html
%
\mode<presentation>
{
  \usetheme{default}      % or try Darmstadt, Madrid, Warsaw, ...
  \usecolortheme{default} % or try albatross, beaver, crane, ...
  \usefonttheme{default}  % or try serif, structurebold, ...
  \setbeamertemplate{navigation symbols}{}
  \setbeamertemplate{caption}[numbered]
} 

\usepackage{polski}

\usepackage[utf8]{inputenc}
\usepackage[T1]{fontenc}
\usepackage{amsfonts}
\usepackage[]{algorithm2e}

\title{Problem podziału na 3 podzbiory}
\author{
       Krzysztof Król\and
                Szymon Dudycz
}
\date{\today}

\begin{document}

\begin{frame}
  \titlepage
\end{frame}

% Uncomment these lines for an automatically generated outline.
%\begin{frame}{Outline}
%  \tableofcontents
%\end{frame}

\section{Opis problemu}

\begin{frame}{Opis problemu}

Zbiór liczb należy podzielić na 3 podzbiory tak, aby:
\begin{itemize}  
\item każda liczba należała do dokładnie jednego podzbioru
\item sumy liczb elementów każdego podzbioru jak najmniej się różniły
\end{itemize}


\end{frame}


\section{Model matematyczny}

\begin{frame}{Model matematyczny}

\textbf{Wejście problemu:}
\\[0.1in]\textbf{Wyjście problemu:}\\


\end{frame}

\section{Górne ograniczenie}

\begin{frame}{Górne ograniczenie}


\end{frame}

\section{Dolne ograniczenie}

\begin{frame}{Dolne ograniczenie}

Jakieś pomysły?

\end{frame}

\section{Algorytm}

\begin{frame}{Algorytm - metaheurystyka}


\end{frame}


\end{document}