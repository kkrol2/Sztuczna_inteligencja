
              
\documentclass{beamer}
%
% Choose how your presentation looks.
%
% For more themes, color themes and font themes, see:
% http://deic.uab.es/~iblanes/beamer_gallery/index_by_theme.html
%
\mode<presentation>
{
  \usetheme{default}      % or try Darmstadt, Madrid, Warsaw, ...
  \usecolortheme{default} % or try albatross, beaver, crane, ...
  \usefonttheme{default}  % or try serif, structurebold, ...
  \setbeamertemplate{navigation symbols}{}
  \setbeamertemplate{caption}[numbered]
} 

\usepackage{polski}

\usepackage[utf8]{inputenc}
\usepackage[T1]{fontenc}
\usepackage{amsfonts}
\usepackage[]{algorithm2e}
\usepackage{lmodern}

\title{Problem podziału na 3 podzbiory}
\author{
       Krzysztof Król\and
                Szymon Dudycz
}
\date{\today}

\begin{document}

\section{Wstęp}
\begin{frame}
  \titlepage
\end{frame}

% Uncomment these lines for an automatically generated outline.
%\begin{frame}{Outline}
%  \tableofcontents
%\end{frame}

\section{Opis problemu}

\begin{frame}{Opis problemu}

Zbiór liczb należy podzielić na 3 podzbiory tak, aby:
\begin{itemize}  
\item każda liczba należała do dokładnie jednego podzbioru
\item sumy liczb elementów każdego podzbioru jak najmniej się różniły
\end{itemize}


\end{frame}


\section{Model matematyczny}

\begin{frame}[t]{Model matematyczny}

\textbf{Wejście problemu:} Dane jest $n$ liczb naturalnych $a_1,\dots,a_n$
\\[0.1in]\textbf{Wyjście problemu:}\\ Dla każdej liczby $a_i$ i zbioru $j$ tworzymy zmienną $x_{i,j}$ oznaczającą czy $a_i$ została przydzielona do zbioru $j$. Wtedy problem można wyrazić za pomocą następującego programu całkowitoliczbowego:
\begin{equation*}
\begin{aligned}
\min &\left| \sum_{i=1}^n x_{i,1} a_i - \sum_{i=1}^n x_{i,2} a_i \right| + \left| \sum_{i=1}^n x_{i,1} a_i - \sum_{i=1}^n x_{i,3} a_i\right| \\
&\qquad\qquad\qquad\qquad\quad\:+ \left| \sum_{i=1}^n x_{i,2} a_i - \sum_{i=1}^n x_{i,3} a_i\right| \\
& x_{i,1} + x_{i,2} + x_{i,3} = 1 \quad\quad \forall i\in \{1,\dots,n\} \\
& x_{i,j} \in \{0,1\} \,\,\,\quad\qquad\qquad \forall i\in \{1,\dots,n\}\quad \forall j\in \{1,2,3\}
\end{aligned}
\phantom{\hspace{6cm}}
\end{equation*}

\end{frame}

\section{Górne ograniczenie}

\begin{frame}{Górne ograniczenie}
Górne ograniczenie można wyznaczyć algorytmem Longest Processing Time. Można udowodnić, że ten algorytm osiąga górne ograniczenie nie gorsze od $11/9*optimal$ dla problemu podziału ciągu, w którym chcemy minimalizować sumę elementów największego podzbioru.\\~\\
\textbf{LPT:}\\
\begin{algorithm}[H]
 $s_1 \gets \emptyset$\;
 $s_2 \gets \emptyset$\;
 $s_3 \gets \emptyset$\;
 $m \gets $ posortowany ciąg wejściowy w nierosnącej kolejności\;
 $iter \gets 0$\;
 \While{$iter < n$}{
    $s \gets min(s_1,s_2,s_3)$\;
    $s \gets s \cup \{m_{iter}\}$\;
    $iter \gets iter+1$\;
  }
\end{algorithm} 

  

\end{frame}

\section{Dolne ograniczenie}

\begin{frame}{Dolne ograniczenie}

Jakieś pomysły?

\end{frame}

\section{Algorytm}

\begin{frame}{Algorytmy}


\begin{itemize}  
\item Algorytm pseudo wielomianowy O($M^2n$) gdzie M to suma elementów ciągu wejściowego
\item Heurystyka Karmakar-Karp
\item Meta-heurystyka - symulowane wyżarzanie
\end{itemize}

\end{frame}


\end{document}